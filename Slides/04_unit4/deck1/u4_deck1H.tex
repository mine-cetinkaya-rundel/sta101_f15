 % -*- TeX-engine: xetex; eval: (auto-fill-mode 0); eval: (visual-line-mode 1); -*-
% Compile with XeLaTeX

%%%%%%%%%%%%%%%%%%%%%%%
% To do before class
%%%%%%%%%%%%%%%%%%%%%%%

% Print off Readiness Assessment 1.
% Send email about registering clicker.
% Test run readiness assessment on iClicker.

%%%%%%%%%%%%%%%%%%%%%%%
% Option 1: Slides: (comment for handouts)   %
%%%%%%%%%%%%%%%%%%%%%%%

%\documentclass[slidestop,compress,mathserif,12pt,t,professionalfonts,xcolor=table]{beamer}
%
%% solution stuff
%\newcommand{\solnMult}[1]{
%\only<1>{#1}
%\only<2->{\red{\textbf{#1}}}
%}
%\newcommand{\soln}[1]{\textit{#1}}

%%%%%%%%%%%%%%%%%%%%%%%
% Option 2: Slides: (to post)                          %
%%%%%%%%%%%%%%%%%%%%%%%

%\documentclass[slidestop,compress,mathserif,12pt,t,professionalfonts,xcolor=table]{beamer}
%
% % solution stuff
% \newcommand{\solnMult}[1]{#1}
% \newcommand{\soln}[1]{}

%%%%%%%%%%%%%%%%%%%%%%%%%%%%%%%
% Option 3: Handouts, without solutions (post before class)    %
%%%%%%%%%%%%%%%%%%%%%%%%%%%%%%%

 \documentclass[11pt,containsverbatim,handout,xcolor=xelatex,dvipsnames,table]{beamer}

 % handout layout
 \usepackage{pgfpages}
 \pgfpagesuselayout{4 on 1}[letterpaper,landscape,border shrink=5mm]

 % solution stuff
 \newcommand{\solnMult}[1]{#1}
 \newcommand{\soln}[1]{}

%%%%%%%%%%%%%%%%%%%%%%%%%%%%%%%%%%%%
% Option 4: Handouts, with solutions (may post after class if need be)    %
%%%%%%%%%%%%%%%%%%%%%%%%%%%%%%%%%%%%

% \documentclass[11pt,containsverbatim,handout,xcolor=xelatex,dvipsnames,table]{beamer}

% % handout layout
% \usepackage{pgfpages}
% \pgfpagesuselayout{4 on 1}[letterpaper,landscape,border shrink=5mm]

% % solution stuff
% \newcommand{\solnMult}[1]{\red{\textbf{#1}}}
% \newcommand{\soln}[1]{\textit{#1}}


%%%%%%%%%%
% Load style file, defaults  %
%%%%%%%%%%

\input{../../lec_style.tex}
% You cannot use numbers when defining variables.  
% Hence the use of letters, A, B, C, etc.

% Course Name
\newcommand{\CourseName}{Sta 101 - Fall 2015}
\newcommand{\InstituteName}{Duke University, Department of Statistical Science}

% Personal Info
\newcommand{\FirstName}{Mine}
\newcommand{\LastName}{\c{C}etinkaya-Rundel}
\newcommand{\OfficeHours}{Tue + Thur 4:30-6pm}
\newcommand{\OfficeHoursLocation}{Old Chem 213}

% Electronic Info
\newcommand{\PersonalSite}{http://stat.duke.edu/~mc301}
\newcommand{\CourseSite}{http://bit.ly/sta101_f15}
\newcommand{\Email}{mine@stat.duke.edu}

% TAs
\newcommand{\TAA}{Erika Ball}
\newcommand{\TAB}{David Clancy}
\newcommand{\TAC}{Reuben McCreanor}
\newcommand{\TAD}{Anne Driscoll}
\newcommand{\TAE}{Megan Robertson}

% Exam Dates
\newcommand{\ExamADate}{Mon, Oct 5}
\newcommand{\ExamBDate}{Mon, Nov 9}
\newcommand{\FinalDate}{Thur, Dec 10 (2-5pm)}


% ALT ALT
% % You cannot use numbers when defining variables.  
% Hence the use of letters, A, B, C, etc.

% Personal Info
\newcommand{\FirstName}{Anthea}
\newcommand{\LastName}{Monod}
\newcommand{\OfficeHours}{TBA}
\newcommand{\OfficeHoursLocation}{TBA}

% Electronic Info
\newcommand{\PersonalSite}{TBA}
\newcommand{\CourseSite}{TBA}
\newcommand{\Email}{TBA}

% TAs
\newcommand{\TAA}{TBA}
\newcommand{\TAB}{TBA}
\newcommand{\TAC}{TBA}
\newcommand{\TAD}{TBA}
\newcommand{\TAE}{TBA}

% Exam Dates
\newcommand{\ExamADate}{TBA}
\newcommand{\ExamBDate}{TBA}
\newcommand{\FinalDate}{TBA}



%%%%%%%%%%%
% Cover slide info    %
%%%%%%%%%%%

\title{Unit 4: Inference for numerical data}
\subtitle{1. Inference using the $t$-distribution}
\author{\CourseName}
\date{}
\institute{\InstituteName}


%%%%%%%%%%%%%%%%%%%%%%%%%
% Begin document and set Helvetica Neue font   %
%%%%%%%%%%%%%%%%%%%%%%%%%

\begin{document}
\fontspec[Ligatures=TeX]{Helvetica Neue Light}

%%%%%%%%%%%%%%%%%%%%%%%%%%%%%%%%%%%

% Title Page

\begin{frame}[plain]

\titlepage

\vfill

{\scriptsize \webLink{\PersonalSite}{Dr. \LastName{}} \hfill Slides posted at  \webURL{\CourseSite}}

\addtocounter{framenumber}{-1} 

\end{frame}

%%%%%%%%%%%%%%%%%%%%%%%%%%%%%%%%%%%%

\section{Housekeeping}

%%%%%%%%%%%%%%%%%%%%%%%%%%%%%%%%%%%%

\begin{frame}
\frametitle{Announcements}

\begin{itemize}

\item 

\end{itemize}

\end{frame}

%%%%%%%%%%%%%%%%%%%%%%%%%%%%%%%%%%%%

\section{Main ideas}

%%%%%%%%%%%%%%%%%%%%%%%%%%%%%%%%%%%%

\subsection{T corrects for uncertainty introduced by plugging in $s$ for $\sigma$}
\label{mi1}

%%%%%%%%%%%%%%%%%%%%%%%%%%%%%%%%%%%%

\begin{frame}
\frametitle{2. T corrects for uncertainty introduced by plugging in $s$ for $\sigma$}

\begin{itemize}

\item CLT says $\bar{x} \sim N(mean = \mu, SE = \frac{\sigma}{\sqrt{n}}$, but, in practice, 
we use $s$ instead of $\sigma$.
\begin{itemize}
\item Plugging in an estimate introduces additional uncertainty.
\item We make up for this by using a more ``conservative" distribution than the normal distribution.
\end{itemize}

\pause

\item Also has a bell shape, but its tails are \hl{thicker} than the normal model's
\begin{itemize}
\item Observations are more likely to fall beyond two SDs from the mean than under the normal distribution.
\end{itemize}

\pause

\item Extra thick tails are helpful for mitigating the effect of a less reliable estimate for the standard 
error of the sampling distribution.

\end{itemize}

\begin{center}
\includegraphics[width=0.5\textwidth]{figures/tDistCompareToNormalDist/tDistCompareToNormalDist}
\end{center}

\end{frame}

%%%%%%%%%%%%%%%%%%%%%%%%%%%%%%%%%%%%

\begin{frame}
\frametitle{T distribution}

\begin{itemize}

\item Always centered at zero, like the standard normal ($z$) distribution

\pause

\item Has a single parameter: \hl{degrees of freedom} (\mathhl{df})
\begin{itemize}
\item one sample: $df = n - 1$
\item two (independent) samples: $df = min(n_1 - 1, n_2 - 1)$
\end{itemize}

\end{itemize}

\begin{center}
\includegraphics[width=0.8\textwidth]{figures/tDistConvergeToNormalDist/tDistConvergeToNormalDist}
\end{center}

\pause

\disc{What happens to shape of the T distribution as $df$ increases?}

\soln{\pause Approaches normal.}

\end{frame}

%%%%%%%%%%%%%%%%%%%%%%%%%%%%%%%%%%%%

\subsection{When comparing means of two groups, ask if paired or independent}
\label{mi2}

%%%%%%%%%%%%%%%%%%%%%%%%%%%%%%%%%%%%

\begin{frame}
\frametitle{2. When comparing means of two groups, ask if paired or independent}

\begin{itemize}

\item dependent (paired) groups (e.g. pre/post weights of subjects in a weight loss study, twin studies, etc.)
\[ SE_{\bar{x}_{diff}} = \frac{s_{diff}}{\sqrt{n_{diff}}} \]

\item independent groups (e.g. grades of students across two sections)
\[ SE_{\bar{x}_1 - \bar{x}_2} = \sqrt{ \frac{s_1^2}{n_1} + \frac{s_2^2}{n_2} } \]

\end{itemize}

\end{frame}

%%%%%%%%%%%%%%%%%%%%%%%%%%%%%%%%%%%%

\begin{frame}
\frametitle{Example 1: Zinc in water}

\disc{Trace metals in drinking water affect the flavor and an unusually high concentration can pose a health hazard. Ten pairs of data were taken measuring zinc concentration in bottom water and surface water at 10 randomly sampled locations.}

\twocol{0.5}{0.5}{
\begin{center}
{\footnotesize
\begin{tabular}{l | c | c}
Location	& bottom	& surface \\
\hline
1&0.43&0.415\\
2&0.266&0.238\\
3&0.567&0.39\\
4&0.531&0.41\\
5&0.707&0.605\\
6&0.716&0.609\\
7&0.651&0.632\\
8&0.589&0.523\\
9&0.469&0.411\\
10&0.723&0.612\\
\end{tabular}
}
\end{center}
}
{
\pause
Water samples collected at the same location, on the surface and in the bottom, cannot be assumed to be independent of each other, hence we need to use a \hl{paired} analysis.
}

\vfill

\ct{Source: \webURL{https://onlinecourses.science.psu.edu/stat500/node/51}}

\end{frame}

%%%%%%%%%%%%%%%%%%%%%%%%%%%%%%%%%%%%

\begin{frame}
\frametitle{Analyzing paired data}

Suppose we want to compare the average zinc concentration levels in the bottom and surface:

\pause

\begin{itemize}

\item When two sets of observations have this special correspondence (not independent), they are said to be \hl{paired}.

\pause

\item To analyze paired data, it is often useful to look at the difference in outcomes of each pair of observations. 

\pause

\item It is important that we always subtract using a consistent order.

\end{itemize}

\pause

\vspace{-0.5cm}

\twocol{0.5}{0.5}{
\begin{center}
{\scriptsize
\begin{tabular}{l | c | c | >{\columncolor[gray]{0.8}} c}
Location	& bottom	& surface & difference\\
\hline
1&0.43&0.415&0.015\\
2&0.266&0.238&0.028\\
3&0.567&0.39&0.177\\
4&0.531&0.41&0.121\\
5&0.707&0.605&0.102\\
6&0.716&0.609&0.107\\
7&0.651&0.632&0.019\\
8&0.589&0.523&0.066\\
9&0.469&0.411&0.058\\
10&0.723&0.612&0.111\\
\end{tabular}
}
\end{center}
}
{
\begin{center}
\includegraphics[width=\textwidth]{figures/zinc/zinc_diff_hist}
\end{center}
}

\end{frame}

%%%%%%%%%%%%%%%%%%%%%%%%%%%%%%%%%%%

\begin{frame}
\frametitle{Parameter and point estimate for paired data}

For comparing average zinc concentration levels in the bottom and surface when the data are paired:

\pause

\begin{itemize}

\item \hl{Parameter of interest:} Average difference between the bottom and surface zinc measurements of \red{all} drinking water.
\[ \mu_{diff} \]

$\:$ \\

\pause

\item \hl{Point estimate:} Average difference between the bottom and surface zinc measurements of drinking water from the \red{sampled} locations.
\[ \bar{x}_{diff} \]

\end{itemize}

\end{frame}

%%%%%%%%%%%%%%%%%%%%%%%%%%%%%%%%%%%

\begin{frame}
\frametitle{Example 2: Gender gap in salaries}

\disc{{\small Since 2005, the American Community Survey polls $\sim$3.5 million households yearly. The following summarizes distribution of salaries of males and females from a random sample of individuals who responded to the 2012 ACS:}}

\begin{center}
%
{\small
\begin{tabular}{lccc}
\hline
			& $\bar{x}$ 	& $s$	& $n$ \\
\hline
male			& 55,890		& 68,767.88	& 470 \\
female		& 29,240		& 32,025.98	& 373 \\
\hline
\end{tabular}
}
%
\includegraphics[width=0.9\textwidth]{figures/acs/sal_gen_box}
\end{center}

\end{frame}

%%%%%%%%%%%%%%%%%%%%%%%%%%%%%%%%%%%%

\begin{frame}

\vfill

ACS: Surge of media attention in spring 2012 when the House of Representatives voted to eliminate the survey. Daniel Webster, Republican congressman from Florida: ``in the end this is not a scientific survey. It's a random survey."

\vfill

\end{frame}

%%%%%%%%%%%%%%%%%%%%%%%%%%%%%%%%%%%%

\begin{frame}

\vfill

\disc{How are the two examples different from each other? How are they similar to each other?}

\vfill

\end{frame}

%%%%%%%%%%%%%%%%%%%%%%%%%%%%%%%%%%%%

\end{document}