% -*- TeX-engine: xetex; eval: (auto-fill-mode 0); eval: (visual-line-mode 1); -*-
% Compile with XeLaTeX

%%%%%%%%%%%%%%%%%%%%%%%
% To do before class
%%%%%%%%%%%%%%%%%%%%%%%

% Print off Readiness Assessment 1.
% Send email about registering clicker.
% Test run readiness assessment on iClicker.

%%%%%%%%%%%%%%%%%%%%%%%
% Option 1: Slides: (comment for handouts)   %
%%%%%%%%%%%%%%%%%%%%%%%

\documentclass[slidestop,compress,mathserif,12pt,t,professionalfonts,xcolor=table]{beamer}

% solution stuff
\newcommand{\solnMult}[1]{
\only<1>{#1}
\only<2->{\red{\textbf{#1}}}
}
\newcommand{\soln}[1]{\textit{#1}}

%%%%%%%%%%%%%%%%%%%%%%%%%%%%%%%
% Option 2: Handouts, without solutions (post before class)    %
%%%%%%%%%%%%%%%%%%%%%%%%%%%%%%%

% \documentclass[11pt,containsverbatim,handout,xcolor=xelatex,dvipsnames,table]{beamer}

% % handout layout
% \usepackage{pgfpages}
% \pgfpagesuselayout{4 on 1}[letterpaper,landscape,border shrink=5mm]

% % solution stuff
% \newcommand{\solnMult}[1]{#1}
% \newcommand{\soln}[1]{}

% % % This breaks things for me for some reason.
% % tell pgfpages how to set page sizes in XeLaTeX
% %\renewcommand\pgfsetupphysicalpagesizes{%
% %   \pdfpagewidth\pgfphysicalwidth\pdfpageheight\pgfphysicalheight%
% %}

%%%%%%%%%%%%%%%%%%%%%%%%%%%%%%%%%%%%
% Option 3: Handouts, with solutions (may post after class if need be)    %
%%%%%%%%%%%%%%%%%%%%%%%%%%%%%%%%%%%%

% \documentclass[11pt,containsverbatim,handout,xcolor=xelatex,dvipsnames,table]{beamer}

% % handout layout
% \usepackage{pgfpages}
% \pgfpagesuselayout{4 on 1}[letterpaper,landscape,border shrink=5mm]

% % solution stuff
% \newcommand{\solnMult}[1]{\red{\textbf{#1}}}
% \newcommand{\soln}[1]{\textit{#1}}

% % % This breaks things for me for some reason.
% % % tell pgfpages how to set page sizes in XeLaTeX
% % \renewcommand\pgfsetupphysicalpagesizes{%
% %    \pdfpagewidth\pgfphysicalwidth\pdfpageheight\pgfphysicalheight%
% % }

%%%%%%%%%%%%%%%%%%%%%%%%%%%%%%%
% Option 4: Notes Only
%%%%%%%%%%%%%%%%%%%%%%%%%%%%%%%

% % See http://tex.stackexchange.com/questions/114219/add-notes-to-latex-beamer
% \documentclass[10pt,containsverbatim,xcolor=xelatex,dvipsnames,table,notes=only]{beamer}

% % handout layout
% \usepackage{pgfpages}
% \pgfpagesuselayout{2 on 1}[letterpaper, landscape, border shrink=5mm]

% % solution stuff
% \newcommand{\solnMult}[1]{#1}
% \newcommand{\soln}[1]{}

% % % Having a problem with this.
% % tell pgfpages how to set page sizes in XeLaTeX
% % \renewcommand\pgfsetupphysicalpagesizes{%
% %   \pdfpagewidth\pgfphysicalwidth\pdfpageheight\pgfphysicalheight%
% %}

%%%%%%%%%%
% Load style file, defaults  %
%%%%%%%%%%

\input{../../lec_style.tex}
% You cannot use numbers when defining variables.  
% Hence the use of letters, A, B, C, etc.

% Course Name
\newcommand{\CourseName}{Sta 101 - Fall 2015}
\newcommand{\InstituteName}{Duke University, Department of Statistical Science}

% Personal Info
\newcommand{\FirstName}{Mine}
\newcommand{\LastName}{\c{C}etinkaya-Rundel}
\newcommand{\OfficeHours}{Tue + Thur 4:30-6pm}
\newcommand{\OfficeHoursLocation}{Old Chem 213}

% Electronic Info
\newcommand{\PersonalSite}{http://stat.duke.edu/~mc301}
\newcommand{\CourseSite}{http://bit.ly/sta101_f15}
\newcommand{\Email}{mine@stat.duke.edu}

% TAs
\newcommand{\TAA}{Erika Ball}
\newcommand{\TAB}{David Clancy}
\newcommand{\TAC}{Reuben McCreanor}
\newcommand{\TAD}{Anne Driscoll}
\newcommand{\TAE}{Megan Robertson}

% Exam Dates
\newcommand{\ExamADate}{Mon, Oct 5}
\newcommand{\ExamBDate}{Mon, Nov 9}
\newcommand{\FinalDate}{Thur, Dec 10 (2-5pm)}


% ALT ALT
% % You cannot use numbers when defining variables.  
% Hence the use of letters, A, B, C, etc.

% Personal Info
\newcommand{\FirstName}{Anthea}
\newcommand{\LastName}{Monod}
\newcommand{\OfficeHours}{TBA}
\newcommand{\OfficeHoursLocation}{TBA}

% Electronic Info
\newcommand{\PersonalSite}{TBA}
\newcommand{\CourseSite}{TBA}
\newcommand{\Email}{TBA}

% TAs
\newcommand{\TAA}{TBA}
\newcommand{\TAB}{TBA}
\newcommand{\TAC}{TBA}
\newcommand{\TAD}{TBA}
\newcommand{\TAE}{TBA}

% Exam Dates
\newcommand{\ExamADate}{TBA}
\newcommand{\ExamBDate}{TBA}
\newcommand{\FinalDate}{TBA}



%%%%%%%%%%%
% Cover slide info    %
%%%%%%%%%%%

\title{Unit 1: Introduction to data}
\subtitle{4. Review of Unit 1}
\author{\CourseName}
\date{}
\institute{\InstituteName}


%%%%%%%%%%%%%%%%%%%%%%%%%
% Begin document and set Helvetica Neue font   %
%%%%%%%%%%%%%%%%%%%%%%%%%

\begin{document}
\fontspec[Ligatures=TeX]{Helvetica Neue Light}

%%%%%%%%%%%%%%%%%%%%%%%%%%%%%%%%%%%

% Title Page

\begin{frame}[plain]

\titlepage

\vfill

{\scriptsize \webLink{\PersonalSite}{Dr. \LastName{}} \hfill Slides posted at  \webURL{\CourseSite}}

\addtocounter{framenumber}{-1} 

\end{frame}

%%%%%%%%%%%%%%%%%%%%%%%%%%%%%%%%%%%

\section{Housekeeping}

%%%%%%%%%%%%%%%%%%%%%%%%%%%%%%%%%%%

\begin{frame}
\frametitle{Announcements}

\begin{itemize}

\item TBA

\end{itemize}

\end{frame}

%%%%%%%%%%%%%%%%%%%%%%%%%%%%%%%%%%%

\section{Be aware of Simpson's paradox}

%%%%%%%%%%%%%%%%%%%%%%%%%%%%%%%%%%%

\begin{frame}
\frametitle{Race and death-penalty sentences in Florida murder cases}

A 1991 study by Radelet and Pierce on race and death-penalty (DP) sentences gives the following table:

\begin{center}
\begin{tabular}{l c c c c}
\hline
Defendant's race    & DP    & No DP     & Total     & \% DP \\
\hline
Caucasian       & 53        & 430   & 483   & \only<2-|handout:0>{\red{11\%}} \\
African American    & 15        & 176   & 191   & \only<3-|handout:0>{\orange{7.9\%}}  \\ 
\hline
Total               & 68        & 606   & 674 
\end{tabular}
\end{center}

\only<4->{
\disc{Who is more likely to get the death penalty?}
}

\vfill

\ct{Adapted from Subsection 2.3.2 of A. Agresti (2002), Categorical Data Analysis, 2nd ed., and \webURL{http://math.stackexchange.com/questions/83756/examples-of-simpsons-paradox}.}

\end{frame}

%%%%%%%%%%%%%%%%%%%%%%%%%%%%%%%%%%%%

\begin{frame}
\frametitle{Another look}

Same data, taking into consideration victim's race:

{\small
\begin{center}
\begin{tabular}{l l c c c c}
\hline
Victim's race       & Defendant's race  & DP    & No DP     & Total     & \% DP \\
\hline
Caucasian       & Caucasian     & 53        & 414   & 467   & \only<2-|handout:0>{\orange{11.3\%}} \\
Caucasian       & African American  & 11        & 37        & 48        & \only<3-|handout:0>{\red{22.9\%}}  \\ 
African American    & Caucasian     & 0     & 16        & 16        & \only<4-|handout:0>{\orange{0\%}}  \\ 
African American    & African American  & 4     & 139   & 143   & \only<5-|handout:0>{\red{2.8\%}}  \\ 
\hline
Total               &               & 68        & 606   & 674 
\end{tabular}
\end{center}
}

\only<6->{
\disc{Who is more likely to get the death penalty?}
}

\end{frame}

%%%%%%%%%%%%%%%%%%%%%%%%%%%%%%%%%%%%

\begin{frame}
\frametitle{Contradiction?}

\begin{itemize}

\item People of one race are more likely to murder others of the same race, murdering a Caucasian is more likely to result in the death penalty, and there are more Caucasian defendants than African American defendants in the sample.

\pause

\item Controlling for the victim's race reveals more insights into the data, and changes the direction of the relationship between race and death penalty.

\pause

\item This phenomenon is called \hl{Simpson's Paradox}: An association, or a comparison, that holds when we compare two groups can disappear or even be reversed when the original groups are broken down into smaller groups according to some other feature (a confounding/lurking variable).

\end{itemize}

\end{frame}

%%%%%%%%%%%%%%%%%%%%%%%%%%%%%%%%%%%

\section{Application exercises}

%%%%%%%%%%%%%%%%%%%%%%%%%%%%%%%%%%%

\begin{frame}
\frametitle{}

\vfill

\app{1.2 Scientific studies in the press}{$\:$\\ See the course website for instructions. \\$\:$}

\vfill

\end{frame}

%%%%%%%%%%%%%%%%%%%%%%%%%%%%%%%%%%%

\begin{frame}[fragile]
\frametitle{}

\vfill

\app{1.3 Histogram to boxplot}{$\:$\\ See the course website for instructions. \\$\:$}

\vfill

\end{frame}

%%%%%%%%%%%%%%%%%%%%%%%%%%%%%%%%%%%

\begin{frame}
\frametitle{}

\vfill

\app{1.4 Randomization testing}{$\:$\\ See the course website for instructions. \\$\:$}

\vfill

\end{frame}

%%%%%%%%%%%%%%%%%%%%%%%%%%%%%%%%%%%%

\end{document}