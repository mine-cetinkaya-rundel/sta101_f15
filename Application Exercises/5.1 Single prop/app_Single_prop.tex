\documentclass[11pt]{article}
\input{../app_style.tex}

%%%%%%%%%%%
% App Ex number    %
%%%%%%%%%%%

% DON'T FORGET TO UPDATE

\newcommand{\appno}[1]
{5.1}

%%%%%%%%%%%%%%
% Turn on/off solutions       %
%%%%%%%%%%%%%%

% Off
\newcommand{\soln}[2]{$\:$\\ \vspace{#1}}{}

%%% On
%\newcommand{\soln}[2]{\textit{\textcolor{custom_red}{#2}}}{}

%%%%%%%%%%%%%%%%
% Document
%%%%%%%%%%%%%%%%

\begin{document}
\fontspec[Ligatures=TeX]{Helvetica Neue Light}

Dr. \c{C}etinkaya-Rundel \hfill Sta 101: Data Analysis and Statistical Inference \\
Duke University - Department of Statistical Science \hfill \\

\ttl{Application exercise \appno{}: \\
Inference for a single proportion}

\inst{$\:$ \\
Team name: \rule{10cm}{0.5pt} \\
$\:$ \\
Lab section: $\qquad$ 8:30 $\qquad$ 10:05 $\qquad$ 11:45 $\qquad$ 1:25 $\qquad$ 3:05 \\
$\:$ \\
Write your responses in the spaces provided below. WRITE LEGIBLY and SHOW ALL WORK! 
Only one submission per team is required. One team will be randomly selected and their 
responses will be discussed and graded. Concise and coherent are best!}

%%%%%%%%%%%%%%%%%%%%%%%%%%%%%%%%%%%%

%\section*{I see dead people!}

\begin{center}
\includegraphics[width=0.5\textwidth]{figures/dead_people.jpg}
\end{center}

Nearly one-in-five U.S. adults (18\%) say they've seen or been in the presence of a ghost, according to a 
2009 Pew Research Center survey. The survey was conducted on 2,003 randomly sampled Americans.

Source: {\footnotesize \url{http://www.pewresearch.org/fact-tank/2013/10/30/18-of-americans-say-theyve-seen-a-ghost/}}

%

Don't forget to check the conditions when completing the following tasks.

\begin{enumerate}

\item You are asked to write a newspaper article about the percentage of people who say they've seen or 
been in the presence of a ghost. You, obviously, shouldn't just report the point estimate from the survey, but instead 
also provide a margin of error around it. Report the findings in one sentence at the 95\% confidence level.

\soln{5cm}{
Conditions: (1) Independence: SRS + 2003 < 10\% of all americans, therefore whether or not one participant in the sample
has seen or been in the presence of ghost is independent of another in the sample. \\
(2) S: $2003 \times 0.18 = 360.54$ and F: $2003 \times 0.82 = 1642.46$, both greater than 10. \\
Therefore we can assume that the sampling distribution of Americans who say they have seen or been in the presence
of a ghost in samples of size 2,003 is nearly normal.\\
$SE = \sqrt{ \frac{ 0.18 \times 0.82 }{ 2003 } } = 0.0086$ \\
$0.18 \pm 1.96 \times 0.0086 = (0.164, 0.196)$ \\
We are 95\% confident that 16.4\% to 19.6\% of Americans have say they've seen or been in the presence of a ghost.
}

\item Gale Weathers, news reporter for a local news program called \textit{Top Story}, wants to report a similar estimate 
for residents her local area of Woodsboro, CA. Gale has no reason to believe that Woodsboro residents are different
from the rest of the US population on the issue of seeing ghosts. She is ok with reporting a margin of error up to 4\%, 
and she is wondering if she can get away with collecting data from a smaller sample. Can you help her out? How many 
people, at a minimum, should Gale sample?

\soln{5cm}{
$0.04 \ge 1.96 \sqrt{\frac{0.18 \times 0.82}{n}} \rightarrow n \ge 52.3$, At least 53.
}

\item Paranormal researcher Dr. Peter Venkman claims that more than 15\% of Americans have seen or been in the 
presence of a ghost. Do these data provide convincing evidence to support Dr. Venkman's claim?

\soln{7cm}{
$H_0: p = 0.15,~H_A: p > 0.15$ \\
Conditions: (1) Independence: Same as before \\
(2) S: $2003 \times 0.15 = 300.45$ and F: $2003 \times 0.85 = 1702.55$, both greater than 10. \\
Therefore we can assume that the sampling distribution of Americans who say they have seen or been in the presence
of a ghost in samples of size 2,003 is nearly normal.\\
$Z = \frac{0.18 - 0.15}{\sqrt{\frac{0.15 \times 0.85}{2003}}} = 3.76$
p-value approx 0, reject $H_0$, the data provide convincing evidence that more than 15\% of Americans have seen 
or been in the presence of a ghost.
}

\item \textbf{Extra credit:} Who are Gale Weathers and Dr. Peter Venkman?

\soln{2cm}{Gale Weathers is the news reporter from \textit{Scream} and Dr. Peter Venkman from \textit{Ghostbusters}.}

\end{enumerate}

%%%%%%%%%%%%%%%%%%%%%%%%%%%%%%%%%%%%

\end{document}